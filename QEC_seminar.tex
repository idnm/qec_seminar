\documentclass[12 pt]{article}
\usepackage[utf8x]{inputenc}
\usepackage{amsmath}
\usepackage{amsfonts}
\usepackage{amssymb}
\usepackage{empheq}
\usepackage[sort&compress,numbers]{natbib}
\usepackage{doi}
\usepackage{esvect}
\usepackage{cancel}
\usepackage{braket}
\usepackage{hyperref}
\hypersetup{
	colorlinks   = true, %Colours links instead of ugly boxes
	urlcolor     = blue, %Colour for external hyperlinks
	linkcolor    = blue, %Colour of internal links
	citecolor   = red %Colour of citations
}

\textheight 23cm
\textwidth 18cm
\voffset=-1.2in
\hoffset= - 0.9in         

\begin{document}
\tableofcontents
\section{Classical linear codes 15.07.22}
\subsection{Repetition code}
Repetition code. Encoding
\begin{align}
0\to000, 1\to111
\end{align}
Decoding: majority vote, e.g. $010\to0, 011\to1$. Succeeds when 0 or 1 bit is flipped, fails when 2 or 3 bits are flipped. Error probability is reduced from $p$ to $3p^2$ (for $p\ll1$).
\subsection{Linear codes}
Repetition code is an example of a linear code. A linear code $(n, k, d)$ is defined by a generator matrix $G$ of size $k\times n$ and a parity check matrix $H$ of size $(n-k)\times n$. Code subspace is $bG$, for all $k$-bitstrings $b$. Alternatively, code supspace consists of all $n$-bitstings $b'$ satisfying $H(b')^T=0$. For the repetition code
\begin{align}
G=\begin{pmatrix}1&1&1\end{pmatrix},\quad H=\begin{pmatrix}1&1&0\\0&1&1\end{pmatrix},\qquad H^T G=0
\end{align}
\subsection{Code distance}
Code distance is the minimum Hamming distance between any two code words. For a linear code, code distance is the minimum (non-zero) weight of $n$-bitstrings in the code space. For the repetition code $d=3$. A code that can correct $t$ errors has distance $d=2t+1$.
\subsection{Dual code}
For a linear code with generator matrix $G$ and parity check matrix $H$ the dual code is another linear code with $G$ and $H$ swapped $G^{\perp}=H, H^{\perp}=G$. For the dual repetition code
\begin{align}
G=\begin{pmatrix}1&1&0\\0&1&1\end{pmatrix},\qquad H=\begin{pmatrix}1&1&1\end{pmatrix},\quad H^TG=0
\end{align}
Dual repetition code encodes 2 bits into 3 bits
\begin{align}
00\to 000\\
01\to 011\\
10\to 110\\
11\to 101
\end{align}
Distance of the code is 2. It can detect one error, but correct none.
\subsection{Hamming bound}
Codes can not be "too good". Encoding $k$-bitstrings into $n$-bitstrings and being able to correct $t$ errors (distance at least $d\ge2t+1$) requires the embedding space to have dimension at least
\begin{align}
2^n\ge 2^k \sum_{i=0}^t C_{n}^i
\end{align}
\subsection{Gilbert–Varshamov bound}
"Good enough" codes exist. Given $n$ physical bits a code with distance $d$ exists encoding $k$ logical bits with
\begin{align}
2^k\ge \frac{2^n}{\sum_{i=0}^{d-1} C_{n}^i}
\end{align}

\section{Quantum repetition code and first look at stabilizer formalism 22.06.2022}
\subsection{Quantum repetition code}
Encoding
\begin{align}
|0\rangle\to \bar{|0\rangle}=|000\rangle, \qquad |1\rangle\to\bar{|1\rangle}\to |111\rangle, \qquad \alpha|0\rangle+\beta |1\rangle\to\alpha\bar{|0\rangle}+\beta\bar{|1\rangle}
\end{align}
\textit{Exercise}: find a unitary quantum circuit that performs the encoding starting from the state $(\alpha|0\rangle+\beta|1\rangle)\otimes|0\rangle\otimes|0\rangle$.
\subsection{X errors}
Correctable errors: $X_1, X_2, X_3$:
\begin{align}
X_1|000\rangle=|100\rangle,\quad X_1|111\rangle=|011\rangle,\quad \dots
\end{align}
\subsection{Syndromes}
Measuring $Z_1Z_2$, $Z_2Z3$ gives syndromes. For correctable errors syndromes are

\begin{tabular}{lll}
	& $Z_1 Z_2$ & $Z_2 Z_3$ \\ \hline
	Id   & 1         & 1         \\ \hline
	$X_1$ & -1        & 1         \\ \hline
	$X_2$ & -1        & -1       \\ \hline
	$X_3$ & 1         & -1       
\end{tabular}

\textit{Excercise}: find syndromes of other errors, e.g. $X_1X_2$ or even $Y_1 Z_3$.

\textit{Excercise}: find a circuit that uses 1 ancilla qubit to measure syndrome $Z_1Z_2$.


\subsection{Measurement}
Measuring $Z$ in state $\alpha|0\rangle+\beta|1\rangle$ gives +1 with probability $|\alpha|^2$ and post-measurement state $|0\rangle$ or $-1$ with probability $|\beta|^2$ and post-measurement state $|1\rangle$. 

\textit{Exercise}: find values, probabilities and post-measurement states of $Z_1 Z_3$ performed on $\alpha|100\rangle+\beta|011\rangle$. What about, say, $X_1$?
\subsection{Necessary and sufficient conditions for error correction}

Let $\{|\bar{i}\rangle\}$ be the code space and $\{E_\alpha\}$ the set of errors. The code can correct these errors iff
\begin{align}
\braket{\bar{i}|E^\dagger_\beta E_\alpha|\bar{j}}=0,\qquad i\neq j\\
\braket{\bar{i}|E^\dagger_\beta E_\alpha|\bar{i}}=C_{\alpha\beta} \quad\text{independent of $i$}
\end{align}
The first condition means no errors can make different logical states overlap (otherwise they could be confused and the errors could not be corrected). The second means that confusing different errors acting on the same state is fine, as long as the correction procedure works identically on all logical states.
\subsection{Shor's code}
Shor's code can correct both $X$ and $Z$ single-qubit errors. It uses 5 qubits. Encoding
\begin{align}
|\bar{0}\rangle=(|000\rangle+|111\rangle)(|000\rangle+|111\rangle)(|000\rangle+|111\rangle)\\
|\bar{1}\rangle=(|000\rangle-|111\rangle)(|000\rangle-|111\rangle)(|000\rangle-|111\rangle)
\end{align}
\textit{Exercise}: propose a syndrome measurement that diagnoses $X_1$ error and a syndrome that diagnoses $Z_1$ error without destroying superposition $\alpha|\bar{0}\rangle+\beta|\bar{1}\rangle$.


\end{document}
