\documentclass[12 pt]{article}
\usepackage[utf8x]{inputenc}
\usepackage{amsmath}
\usepackage{amsfonts}
\usepackage{amssymb}
\usepackage{empheq}
\usepackage[sort&compress,numbers]{natbib}
\usepackage{doi}
\usepackage{esvect}
\usepackage{cancel}
\usepackage{braket}
\usepackage{hyperref}
\hypersetup{
	colorlinks   = true, %Colours links instead of ugly boxes
	urlcolor     = blue, %Colour for external hyperlinks
	linkcolor    = blue, %Colour of internal links
	citecolor   = red %Colour of citations
}

\textheight 23cm
\textwidth 18cm
\voffset=-1.2in
\hoffset= - 0.9in         

\begin{document}
\section{Classical linear codes 15.07.22}
\subsection{Repetition code}
Repetition code. Encoding
\begin{align}
0\to000, 1\to111
\end{align}
Decoding: majority vote, e.g. $010\to0, 011\to1$. Succeeds when 0 or 1 bit is flipped, fails when 2 or 3 bits are flipped. Error probability is reduced from $p$ to $3p^2$ (for $p\ll1$).
\subsection{Linear codes}
Repetition code is an example of a linear code. A linear code $(n, k, d)$ is defined by a generator matrix $G$ of size $k\times n$ and a parity check matrix $H$ of size $(n-k)\times n$. Code subspace is $bG$, for all $k$-bitstrings $b$. Alternatively, code supspace consists of all $n$-bitstings $b'$ satisfying $H(b')^T=0$. For the repetition code
\begin{align}
G=\begin{pmatrix}1&1&1\end{pmatrix},\quad H=\begin{pmatrix}1&1&0\\0&1&1\end{pmatrix},\qquad H^T G=0
\end{align}
\subsection{Code distance}
Code distance is the minimum Hamming distance between any two code words. For a linear code, code distance is the minimum (non-zero) weight of $n$-bitstrings in the code space. For the repetition code $d=3$. A code that can correct $t$ errors has distance $d=2t+1$.
\subsection{Dual code}
For a linear code with generator matrix $G$ and parity check matrix $H$ the dual code is another linear code with $G$ and $H$ swapped $G^{\perp}=H, H^{\perp}=G$. For the dual repetition code
\begin{align}
G=\begin{pmatrix}1&1&0\\0&1&1\end{pmatrix},\qquad H=\begin{pmatrix}1&1&1\end{pmatrix},\quad H^TG=0
\end{align}
Dual repetition code encodes 2 bits into 3 bits
\begin{align}
00\to 000\\
01\to 011\\
10\to 110\\
11\to 101
\end{align}
Distance of the code is 2. It can detect one error, but correct none.
\subsection{Hamming bound}
"Good enough" codes exist. Encoding $k$-bitstrings and being able to correct $t$ errors (distance at least $d\ge2t+1$) requires the embedding space to have dimension at least $n$
\begin{align}
2^n\ge 2^k \sum_{i=0}^t C_{n}^i
\end{align}
\subsection{Gilbert–Varshamov bound}
Codes need not be "too bad". Given $n$ physical bits a code  with distance $d$ exists encoding $k$ logical bits with
\begin{align}
2^k\ge \frac{2^n}{\sum_{i=0}^{d-1} C_{n}^i}
\end{align}
\end{document}
